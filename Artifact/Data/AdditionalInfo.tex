\documentclass{article}

% ? This document exists for me to impute custom info to the vector DB.

% ? By default, none of the policies state things like "this is BCU located on Curzon Street", 
% ? so the chatbot doesn't necessarily know that info.

% ? This also allows the imputation of extra data not directly stored as PDFs on the site, such as
% ? society-related info.

% ! Society info is taken from the BCUSU site.

\begin{document}

\section{The university}
Birmingham City University (BCU) is one of many universities located in Birmingham. It is split over two main campuses:
"City Centre Campus" located at postcode B4 7BD, and "City South Campus" located at postcode B15 3TN.

The City Centre campus consists of the following buildings: the "Curzon Building" located at 4 Cardigan Street,
"Millennium Point" located on Curzon Street, "Parkside Building" located at 5 Cardigan Street, "Joseph Priestley Building"
located at 6 Cardigan Street, and "STEAMHouse" located at Belmont Row.

The City South campus consists of the "Seacole Building", located at \\ Westbourne Road.

\section{Student Union}
BCU also has a Student Union, known as "BCUSU" or sometimes called the "SU". BCUSU are there to improve students' university experience and help them 
have the best possible time during their university tenure. They are an official independent charity who represent student 
issues and concerns, and liaise with the university to make positive change. BCUSU is democratically run by students, with elections 
hosted annually for SU positions.

BCUSU offer three main things to help university students: Advice from professionally trained advisers on topics like money, health, academia 
and housing. Representation to allow all voices to be heard, and Societies for students to get involved with and befriend like-minded people 
with similar interests.

% ! Insert who the new officers are, and what the positions entail, once the elections finish.

\section{Societies}
Societies are a large part of BCU, where students can find new friends with similar interests. There are 80 active societies as of 
February 2025 divided over multiple categories. These categories are: "General Interest and Media", "Academic", "Faith and Culture", "Liberation", 
"Active and Performance", and "Sports". 

% ? I didn't realise how large of an undertaking this would be.

Most societies have paid memberships of varying cost, excluding Liberation societies which are instead funded directly by the SU and have no 
cost to join.

\end{document}