\documentclass[12pt]{report}
\usepackage{graphicx} % Required for inserting images
\usepackage[a4paper, margin=2.5cm]{geometry}
\graphicspath{{../.images/}}

\title{CMP6200 Individual Undergraduate Project}
\author{Lewis Higgins - Student ID 22133848}


\usepackage[utf8]{inputenc}
\usepackage[T1]{fontenc}
\usepackage{float} % here for H placement parameter
\usepackage{subcaption}

\usepackage{filecontents}
\usepackage[
    firstinits=true, % render first and middle names as initials
    useprefix=true,
    maxcitenames=3,
    maxbibnames=99,
    style=authoryear,
    dashed=false, % re-print recurring author names in bibliography
    natbib=true,
    url=false,
    sorting=none
]{biblatex} % biblatex config for harvard refs

\renewbibmacro*{volume+number+eid}{%
    \printfield{volume}%
%  \setunit*{\adddot}% DELETED
    \setunit*{\addnbspace}% NEW (optional); there's also \addnbthinspace
    \printfield{number}%
    \setunit{\addcomma\space}%
    \printfield{eid}}
\DeclareFieldFormat[article]{number}{\mkbibparens{#1}}

\DeclareLabeldate{\field{date}\field{eventdate} \field{origdate}\literal{nodate}}

\addbibresource{proposal.bib}

% Use single quotes around titles:
\usepackage[british]{babel}
\usepackage{csquotes}

\usepackage{hyperref}

\hypersetup{
    colorlinks=true,
    linkcolor=black,
    filecolor=magenta,
    urlcolor=blue,
    citecolor=black,
}


\urlstyle{same}


% To prevent "Chapter N" display for each chapter
\usepackage[compact]{titlesec}
\usepackage{wasysym}
\usepackage{import}

\titlespacing*{\chapter}{0pt}{-2cm}{0.5cm}
\titleformat{\chapter}[display]
{\normalfont\bfseries}{}{0pt}{\Huge}

\newcommand\blfootnote[1]{
    \begingroup
    \renewcommand\thefootnote{}\footnote{#1}
    \addtocounter{footnote}{-1}
    \endgroup
}

\usepackage{fancyhdr}
\usepackage{calc}
\pagestyle{fancy}

\usepackage{tcolorbox}

\setlength\headheight{37pt}

\renewcommand{\chaptermark}[1]{%
    \markboth{#1}{}}

\lhead{Lewis Higgins - ID 22133848~~~~~~~~~~~~~~~\includegraphics[width=1.75cm]{bcu logo}}
\fancyhead[R]{\leftmark}

\usepackage{xcolor} % Might be useful


\begin{document}

    \makeatletter
    \begin{titlepage}
        \includegraphics[width=0.3\linewidth]{BCUWide.jpg}\\[4ex]
        \vspace{1cm}
        \begin{center}
            {\huge \bfseries  CMP6200}\\[2ex]
            {\huge \bfseries  Individual Undergraduate Project}\\[2ex]
            {\huge \bfseries 2024 - 2025}\\[6ex]
            {\large \bfseries A1 - Proposal}\\[10ex]
            {\huge \bfseries University Artifically Intelligent Assistant}\\[6ex]
            \includegraphics[width=0.1\linewidth]{Symbol.png}\\[50ex]
            Course: Computer \& Data Science\\
            Student Name: Lewis Higgins\\
            Student Number: 22133848\\
            Supervisor Name: Dr. Atif Azad
        \end{center}
    \end{titlepage}
    \makeatother
    \thispagestyle{empty}
    \newpage

    \tableofcontents
    %\footnotesize{\listoffigures}

    \chapter{Introduction}\label{ch:introduction}
    \section{Background and Rationale}
    With artificial intelligence (AI) becoming increasingly more powerful and useful in 
    recent years, in some cases even surpassing humans in areas like language and image 
    recognition~\autocite{owid-artificial-intelligence}. It is therefore essential that 
    higher education institutions take advantage of it and ensure to keep up-to-date with its
    developments in the interest of academic integrity, which was also stated in the 
    Higher Education Policy Report dated 28 March 2024 - "Higher education will have to adopt,
    adapt, collaborate and lead to take advantage of AI while managing the risks"~\autocite{AIUni}.\\
    
    \noindent This project aims to leverage recent technical developments in natural language processing (NLP)
    to create a digital assistant for university life that students can use to gain information on topics
    such as university policies and locations on campus. This is because attending university for the first time
    is a daunting and stressful experience for many, often due to it being a new and unfamiliar environment 
    where students have full independence unlike their previous educational settings, which could possibly 
    lead to declines in academic performance and social activity. Some students may not 
    wish to speak with newer people who they don't know about university topics for fear of 
    ridicule or embarrassment, and would benefit from a digital companion to help them to become 
    acquainted with their new environment without the perceived risk of social judgement.

    

    \section{Key Themes/Topics}
    AAAAAA

    \chapter{Aims and Objectives}
    \section{Project Aim}
    AAAAAA
    \section{Project Objectives}

    \chapter{Project Planning}
    \section{Initial Project Plan}
    AAAAAA
    \section{Resources}
    AAAAAA
    \section{Risk Assessments}
    AAAAAA

    \chapter{Project Review and Methodology}
    \section{Critique of Past Similar Projects}
    AAAAAA
    \section{Literature Search Methodology}
    AAAAAA
    \section{Initial Literature Search Results}
    AAAAAA

    \addcontentsline{toc}{chapter}{Bibliography}
    \printbibliography

\end{document}