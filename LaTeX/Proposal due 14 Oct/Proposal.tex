\documentclass[12pt]{report}
\usepackage{graphicx} % Required for inserting images
\usepackage[a4paper, margin=2.5cm]{geometry}
\graphicspath{{../.images/}}

\title{CMP6200 Individual Undergraduate Project}
\author{Lewis Higgins - Student ID 22133848}


\usepackage[utf8]{inputenc}
\usepackage[T1]{fontenc}
\usepackage{float} % here for H placement parameter
\usepackage{subcaption}

\usepackage{filecontents}
\usepackage[
    firstinits=true, % render first and middle names as initials
    useprefix=true,
    maxcitenames=3,
    maxbibnames=99,
    style=authoryear,
    dashed=false, % re-print recurring author names in bibliography
    natbib=true,
    url=false,
    sorting=none
]{biblatex} % biblatex config for harvard refs

\renewbibmacro*{volume+number+eid}{%
    \printfield{volume}%
%  \setunit*{\adddot}% DELETED
    \setunit*{\addnbspace}% NEW (optional); there's also \addnbthinspace
    \printfield{number}%
    \setunit{\addcomma\space}%
    \printfield{eid}}
\DeclareFieldFormat[article]{number}{\mkbibparens{#1}}

\DeclareLabeldate{\field{date}\field{eventdate} \field{origdate}\literal{nodate}}

\addbibresource{proposal.bib}

% Use single quotes around titles:
\usepackage[british]{babel}
\usepackage{csquotes}

\usepackage{hyperref}

\hypersetup{
    colorlinks=true,
    linkcolor=black,
    filecolor=magenta,
    urlcolor=blue,
    citecolor=black,
}


\urlstyle{same}


% To prevent "Chapter N" display for each chapter
\usepackage[compact]{titlesec}
\usepackage{wasysym}
\usepackage{import}

\titlespacing*{\chapter}{0pt}{-2cm}{0.5cm}
\titleformat{\chapter}[display]
{\normalfont\bfseries}{}{0pt}{\Huge}

\newcommand\blfootnote[1]{
    \begingroup
    \renewcommand\thefootnote{}\footnote{#1}
    \addtocounter{footnote}{-1}
    \endgroup
}


%\lhead{Lewis Higgins - ID 22133848~~~~~~~~~~~~~~~\includegraphics[width=1.75cm]{bcu logo}}
%\fancyhead[R]{\leftmark}

\usepackage{xcolor} % Might be useful

\usepackage{colortbl}
\usepackage{longtable}
\usepackage{amssymb}

\begin{document}

    \makeatletter
    \begin{titlepage}
        \includegraphics[width=0.3\linewidth]{BCUWide.jpg}\\[4ex]
        \vspace{1cm}
        \begin{center}
            {\huge \bfseries  CMP6200}\\[2ex]
            {\huge \bfseries  Individual Undergraduate Project}\\[2ex]
            {\huge \bfseries 2024 - 2025}\\[6ex]
            {\large \bfseries A1 - Proposal}\\[10ex]
            {\huge \bfseries University Artifically Intelligent Assistant}\\[6ex]
            \includegraphics[width=0.1\linewidth]{Symbol.png}\\[50ex]
            Course: Computer \& Data Science\\
            Student Name: Lewis Higgins\\
            Student Number: 22133848\\
            Supervisor Name: Dr. Atif Azad
        \end{center}
    \end{titlepage}
    \makeatother
    \thispagestyle{empty}
    \newpage

    \tableofcontents
    %\footnotesize{\listoffigures}

    \chapter{Introduction}\label{ch:introduction}
    \section{Background and Rationale}
    With artificial intelligence (AI) becoming increasingly more powerful and useful in 
    recent years, in some cases even surpassing humans in areas like language and image 
    recognition~\autocite{owid-artificial-intelligence}. It is therefore essential that 
    higher education institutions take advantage of it and ensure to keep up-to-date with its
    developments in the interest of academic integrity, which was also stated in the 
    Higher Education Policy Report dated 28 March 2024 - "Higher education will have to adopt,
    adapt, collaborate and lead to take advantage of AI while managing the risks"~\autocite{AIUni}.\\
    
    \noindent This project aims to leverage recent technical developments in natural language processing (NLP)
    to create a digital assistant for university life that students can use to gain information on topics
    such as university policies and locations on campus. This is because attending university for the first time
    is a daunting and stressful experience for many, often due to it being a new and unfamiliar environment 
    where students have full independence unlike their previous educational settings, which could possibly 
    lead to declines in academic performance and social activity. Some students may not 
    wish to speak with newer people who they don't know about university topics for fear of 
    ridicule or embarrassment, and would benefit from a digital companion to help them to become 
    acquainted with their new environment without the perceived risk of social judgement.

    

    \section{Key Themes/Topics}
    This project undertakes the following key themes:

    \begin{itemize}
        \item Natural Language Processing (NLP) 
            \begin{itemize}
                \item As the backbone of this project, extensive research into this topic will be necessary 
                to ensure users have a smooth experience.
            \end{itemize}
        \item Embedding models
            \begin{itemize}
                \item To store non-numeric data such as university policies, a suitable embedding model
                will be necessary to vectorise said data into a numerical representation interpretable by
                the machine learning model.
            \end{itemize}
    \end{itemize}

    \chapter{Aims and Objectives}
    \section{Project Aim}
    This project aims to aid new and existing students alike while they are attending university with 
    helpful information about the university itself, such as university societies, locations/campuses, 
    and policies through the medium of a digital chatbot companion.

    \section{Project Objectives}
    \begin{itemize}
        \item \large \textbf{WIP.}
    \end{itemize}


    \chapter{Project Planning}
    \section{Initial Project Plan}
    \begin{enumerate}
        \item Research
        \begin{enumerate}
            \item Conduct heavy research into machine learning and natural language processing to bolster my 
            knowledge of the topics to assist in the development of the chatbot.
            \item Identify similar projects that already exist to understand where challenges may arise in 
            development and how to differentiate my work to make it stand out and provide unique value.
        \end{enumerate}

        \item Data collection
        \begin{enumerate}
            \item To present users with information from the university, I must first collect this information
            for myself from sources such as the university website and the Student Union.
        \end{enumerate}

        \item \large \textbf{Section WIP.}
    \end{enumerate}

    \section{Resources}
    \begin{itemize}
        \item An integrated development environment (IDE) for Python
        \begin{itemize}
            \item Visual Studio Code is a lightweight editor that supports most programming languages,
            including Python.
            \item An alternative could be JetBrains' PyCharm Professional, which I can access at no charge due
            to being a student.
        \end{itemize}

        \item Machine learning libraries \& frameworks
        \begin{itemize}
            \item Examples include PyTorch, TensorFlow and SpaCy.
        \end{itemize}

        \item A powerful computer.
        \begin{itemize}
            \item Training machine learning models requires significant processing power and RAM. I own a decently 
            powerful computer with a higher-end graphics card which should be able to handle a project of
            this scale.
        \end{itemize}

        \item A platform for the chatbot.
        \begin{itemize}
            \item Many messaging services allow developers to add bots, such as Facebook Messenger, WhatsApp or
            Discord.
        \end{itemize}

    \end{itemize}

    \section{Risk Assessments}
    \begin{table}[H]
        \centering
        \begin{tabular}{ |p{0.3\textwidth}|p{0.12\textwidth}|p{0.1\textwidth}|p{0.1\textwidth}|p{0.25\textwidth}|}
            \hline
            \cellcolor{blue!25}Risk & \cellcolor{blue!25}Likelihood  &
            \cellcolor{blue!25}Severity & \cellcolor{blue!25}Overall & \cellcolor{blue!25}Mitigation\\
            \hline

            The devices used to write code and documentation could encounter hard drive corruption, potentially 
            losing considerable amounts of work. & 2 & 5 & Medium & Ensure that all work is regularly backed up to 
            the cloud and/or a secondary device. Github will be used as a version control system for the project 
            to keep an audit trail of all changes.\\
           
            \hline

            Time constraints could potentially cause rushed and poor-quality work if development is not to a 
            consistent and regular schedule balanced with other university modules. & 3 & 4 & Medium-High &
            Ensure that work is completed at regular intervals, even if the amount at each interval is small. 
            In doing so, it will be much easier to balance three simultaneous deadlines. \\

            \hline

            Personal circumstances could cause progress to slow or halt if something unexpected were to occur
            during the year. & 3 & 4 & Medium-High & Try to be ahead of deadlines where possible to ensure that
            there is free time that could be used in the event of a sudden break becoming necessary.\\
            
            \hline

            Budget constraints could be an issue during development. & 2 & 2 & Low & Use open-source or lower
            cost resources during development, and create a budget to adhere to. \\ 
            
            \hline

            

        \end{tabular}\label{tab:risks}
    \end{table}

    \chapter{Project Review and Methodology}
    \section{Critique of Past Similar Projects}
    AAAAAA
    \section{Literature Search Methodology}
    AAAAAA
    \section{Initial Literature Search Results}
    AAAAAA

    \addcontentsline{toc}{chapter}{Bibliography}
    \printbibliography

\end{document}