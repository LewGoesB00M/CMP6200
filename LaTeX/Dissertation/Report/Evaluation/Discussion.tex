\section{Discussion}

\subsection{Accomplishments}
Overall, it is safe to say that the project can be considered a success, though it certainly is not without flaw.
It was previously mentioned that the chatbot had an accuracy of 80\% according to GEval, though this was only against a dataset 
of 10 questions. It would be much more suitable to expand this training dataset, alongside gathering actual user feedback.

\para Additionally, my own limitations in knowledge when it came to LangChain and LangGraph meant that I was unable to optimally refine the 
retrieval tool to work on the failed questions in time, and I was also unable to successfully implement a ReAct agent as researched in the 
literature review.

\para Despite the project's few failures, there was a greater amount of successes. I have hugely increased my own knowledge of Python, LLMs,
and RAG. These are three critical skills to have if I intend to work in software development due to recent trends with companies 
becoming more reliant on LLMs. 

\para Furthermore, through developing the chatbot and performing the extensive research required, I believe my skills as an overall software 
developer have enhanced in a way that is not exclusive to Python; I have become much more aware of how to interpret API references and documentation,
meaning that my ability to adapt to new tech stacks as seen in this project should now be a much faster process.

\para The achievement I am most proud of with the chatbot is the cost-saving effort of the conditional branches. Originally, the chatbot would 
query the database for every prompt given, even if it was something as simple as 'Hello!'. This would lead to 6,000 characters worth of university
data which would not be relevant to the query being given to the LLM, wasting processing time and money through the greatly increased token cost 
of such a prompt, as well as resulting in a strange and irrelevant response that would only serve to confuse the user. 

\section{Development process}\label{sec:EvalProcess}
\subsection{Positives}
All functional requirements stated in Section \ref{sec:Requirements} for the final product, as well as the original aims and objectives 
of the project, were successfully met with a working chatbot with good accuracy on BCU-related topics being produced in a timely fashion.

\para Furthermore, with the project being a solo endeavour, a comprehensive understanding of the project management life cycle was 
obtained from conception to completion. As a result, I believe my problem-solving and decision-making skills have greatly improved.   

\subsection{Negatives}
As identified previously in Section \ref{sec:Limitations}, the most significant limitation throughout the development process 
was the amount of time available. Over the course of the project's development, significant extenuating circumstances occurred 
leading to the lack of some desired features and lower quality of others. Furthermore, balancing the production of this project 
alongside four other university modules simultaneously proved to be an arduous task that I was unable to efficiently solve to 
a level I would have preferred.

\para Cost proved to be a much lesser restriction than initially anticipated, due to the cost efficiency 
provided through the identification of OpenAI's lower-end models through thorough research. 

\para However, the other limitations specified in Section \ref{sec:Limitations} also played key roles of their own, though less significant 
than the time restrictions. Most notably of these was my own lack of experience with LLMs. Developing a product using a tech stack 
I was entirely unfamiliar with prior to development proved to be highly difficult.