\chapter{Conclusions}
% \textbf{"You should not include any new information or discussion in this section."}
% This section must also link to the project's objectives.

In conclusion, this report has thoroughly detailed the design and development process of a chatbot aimed to assist new university students 
to become acclimated to their new surroundings through an interactive conversational medium to learn key information about the university 
such as its governing policies.

\para A thorough literature review on surrounding topics and the current state-of-the-art was conducted to allow for the chatbot's development
and documentation of the development process and final product. The final product itself performed to a suitable degree, answering 80\% of its 
evaluation golden dataset questions correctly in relation to its expected answers, though improvements perhaps could have been made to increase this 
percentage.

\para Multiple baseline systems were tested to ensure that the chatbot was using the optimal vector store to answer questions with the highest 
accuracy. This was verified using DeepEval's G-Eval metric paired with manual inspection where the metric was occasionally incorrect due to the 
use of an LLM judge. Of the four FAISS vector stores accessible by the chatbot, the one with the largest chunk size and chunk overlap performed 
the best.

\para The final product performs well, meeting and in some cases exceeding its stated functional requirements, though was not 
able to meet all of its non-functional requirements. Despite these minor shortcomings, the chatbot is functional and a very viable source of 
BCU-related information using natural language prompts for specific information rather than having to manually consult long policies and web searches. 


\nocite{projectGithub} % Invisible cite.