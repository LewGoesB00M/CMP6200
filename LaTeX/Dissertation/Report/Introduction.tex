\chapter{Introduction}


\section{Problem definition}
New university students often face challenges in acclimating to their new university life \autocite{oxforduniversitySupportingAcademicTransition}.
These challenges can stem from difficulty in locating buildings or understanding university policies, and traditional support systems, 
such as printed materials or static websites, can often be insufficient, as they require students to navigate complex 
and scattered resources. Additionally, relying on human staff for queries can become especially difficult during 
peak times, such as the start of the academic year.


\section{Aim}
This project aims to aid new and existing students alike while they are attending university with 
helpful information about the university itself, such as general university information, locations/campuses, 
and policies through the medium of a digital chatbot companion to converse with.

\section{Scope}
This project will focus on the development of a chatbot with a knowledge base of university data to query and retrieve information from.
This will be accomplished by using an OpenAI LLM via their API, which will be customised to enable Retrieval-Augmented Generation (RAG),
a technology which enhances an LLM's knowledge by giving it access to additional information that was not originally present in their training data
\autocite{lewis_retrieval-augmented_2021}, making it highly contextualised and usable for BCU-related queries.

\para The chatbot will have a graphical user interface (GUI) to ensure that users can quickly come to terms with how to use the chatbot,
which will be hosted as a web application accessible on the local network. Creating a hosted website is not within this project's scope as this 
would introduce unnecessary cost and waste time.

\para This project will also involve the testing and evaluation of the chatbot to ensure its usability and accuracy with a variety of BCU-related 
questions.

\section{Rationale}
The rapid development of artificial intelligence (AI), particularly in natural language processing (NLP) and large language models (LLMs), provides 
an opportunity to address these challenges. Therefore, the problem also lies in creating a reliable, cost-effective, and user-friendly digital 
assistant capable of answering various university-related inquiries with a high level of accuracy, which this project 
aims to solve. Unlike human staff, a chatbot can theoretically run at all times and respond at a much faster rate, 
meaning students could get the information they need at any time in real-time.

\para Many LLMs already exist, such as ChatGPT \autocite{openaiChatGPT}, DeepSeek \autocite{deepseekDeepSeek}, and 
Gemini \autocite{googleGeminiChatSupercharge}. However, these LLMs do not possess specific BCU-related knowledge. 
This provides a significant opportunity for the assistance of BCU students, which this project aims to capitalise on.
% Therefore, 
% this project will utilise RAG to enhance an existing LLM with topical knowledge to produce 
% a functional chatbot which students will be able to query on BCU-related topics.

\section{Objectives}\label{sec:AimsAndObjectives}
The project's objectives are to:

\begin{itemize}
    \item Conduct a thorough literature review on the surrounding topics, especially AI, LLMs and NLP.
    \item Create effective documentation for all stages of development, highlighting challenges faced during the process.
    \item Leverage Retrieval-Augmented Generation alongside a cloud-based LLM to query a vector database of university-related data.
    \item Develop a chatbot capable of accurately answering user queries related to university 
    buildings, policies, and general information with a minimum 75\% accuracy rate.
    \item Evaluate the effectiveness of an AI assistant on university student acclimatization.
\end{itemize}

% \section{Background information}
% Possibly unnecessary?